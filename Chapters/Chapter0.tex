\chapter*{مقدمـــة} % 
\label{Chapter0} 
\renewcommand{\theequation}{\arabic{chapter}.\arabic{equation}}
%----------------------------------------------------------------------------------------

% Define some commands to keep the formatting separated from the content 
\newcommand{\keyword}[1]{\textbf{#1}}
\newcommand{\tabhead}[1]{\textbf{#1}}
\newcommand{\code}[1]{\texttt{#1}}
%\newcommand{\file}[1]{\texttt{\bfseries#1}}
%\newcommand{\option}[1]{\texttt{\itshape#1}}

%----------------------------------------------------------------------------------------

طُور ميكانيك الكم (QM) على مدار العديد من العقود؛ إذ بدأ كمجموعة من التفسيرات الرياضية الجدلية للتجارب التي لم تتمكن الرياضيات الكلاسيكية من شرحها. بدأ هذا القسم من الفيزياء مع بداية القرن العشرين –في نفس الوقت الذي نشر فيه البرت اينشتاين نظريته في النسبية العامة وهي وصف رياضي ومختلف في الفيزياء حيث يصف حركة الأشياء عند السرعات الكبيرة جداً. على النقيض من النسبية، فإن ميكانيك الكم لا يُمكن نسبها إلى عالم واحد؛ فقد ساهم العديد من العلماء في تأسيس المبادئ لميكانيك الكم والتي حصلت بشكلٍ تدريجي على القبول والتأكيد التجريبي بين الأعوام 1900 و1930؛ و من هذه المبادئ:
\begin{enumerate}
	\item 


الخواص المُكممة (Quantized properties): يُمكن لخواص محددة، مثل الموقع والسرعة واللون، في بعض الأحيان أن تكون محددة الكميات بشكلٍ مشابه تماماً لإجراء اتصال ما بالانتقال من "الضغط" على رقم إلى الضغط على آخر. وتحدى هذا الأمر الافتراض الأساسي للميكانيك الكلاسيكي والذي ينص على أنه يجب على مثل هذه الخواص أن تُوجد على شكل طيف سلس ومستمر. ولوصف فكرة بعض تلك الخواص المشابهة لعملية الضغط عند الاتصال، صاغ العلماء كلمة "مكممة – quantized".

	\item 
جسيمات الضوء (Particles of light): يُمكن للضوء أن يتصرف في بعض الأحيان على شكل جسيم. تم استقبال هذا الأمر في البداية بانتقادات لاذعة كونه يتناقض مع 200 عام من التجارب التي بينت أن الضوء يتصرف كموجة –بشكلٍ مشابه كثيراً للتموجات الموجودة على سطح بحيرة هادئة. يتصرف الضوء بشكلٍ مشابه عندما يرتد عن جدار وينحني حول الزوايا ويوجد أيضاً قمم واختناقات يُمكن أن تقوم بإفناء بعضها في هذه الموجة أو تعزيز بعضها؛ فإضافة قمم موجية إلى بعضها يؤدي إلى الحصول على ضوء أشد لمعاناً، في حين تُنتج الأمواج التي تُفني بعضها الظلمة. يُمكن التفكير بمصدر الضوء على أنه كرة موجودة على عصا تقوم بالاهتزاز بشكل منسجم في مركز بحيرة ما؛ وتعود الألوان الصادرة إلى المسافة الموجودة بين قمتين وهو أمر يُحدده سرعة إيقاع الكرة.

	\item 
أمواج المادة (Waves of matter): يُمكن للمادة أيضاً أن تسلك سلوك موجة. ويتعارض هذا الأمر مع 30 عام من التجارب التي تُبين أن المادة (مثل الالكترونات) تُوجد على شكل جسيمات. الخوّاص المُكممة في العام 1900، بحث الفيزيائي الألماني ماكس بلانك في كيفية تفسير توزع الألوان الصادرة على طول الطيف الخاص بتوهج الأجسام الساخنة-الحمراء والساخنة-البيضاء –مثل أسلاك المصابيح الضوئية. وعندما أراد مناقشة المعنى المادي للمعادلة، اشتق معادلةً لوصف ذلك التوزع؛ وأدرك بلانك أنها تتضمن اجتماع لألوان محددة فقط (بصرف النظر عن عددها الكبير) يتم إصدارها من قبل الجسم. بطريقة أو بأخرى، الألوان مُكممة! كان ذلك الأمر غير متوقع لأن الضوء فُهم على أنه يتصرف كموجة، ما يعني أن قيم الألوان يجب أن تكون عبارة عن طيف مستمر. ما هو الشيء الذي أمكنه منع الذرات من إنتاج الألوان الموجودة على طول مضاعفات تلك الأعداد؟ بدا الأمر غريباً لبلانك بحيث أنه اعتبر أن التكميم (quantization) ما هو إلا خدعة رياضية. ويقول هيلج كراخ (Helge Kragh) في مقالته عام 2000 في مجلة عالم الفيزياء (ماكس بلانك، الثوري المتردد) "إذا ما حصلت ثورة في الفيزياء في ديسمبر/كانون الأول عام 1900، فلا أحد قد لاحظها. وبلانك لم يكن استثناءً...". احتوت معادلة بلانك أيضاً على عدد سيصبح لاحقاً مهم جداً في التطوير المستقبلي لـ QM؛ ويُعرف اليوم بـ "ثابت بلانك". ساعد التكميم على شرح ألغاز أخرى في الفيزياء. ففي العام 1907، استخدم اينشتاين فرضيات بلانك الخاصة بالتكميم لشرح السبب الكامن وراء تغير درجة حرارة المواد الصلبة بمقادير مختلفة إذا ما قمت بتزويدها بنفس الكمية من الحرارة ولكن غيرت من درجة الحرارة الابتدائية.
\end{enumerate} 
منذ أوائل القرن التاسع عشر، أظهر علم التحليل الطيفي أن العناصر المختلفة تقوم بإصدار وامتصاص ألوان محددة من الضوء وتُعرف تلك الألوان بـ "الخطوط الطيفية". على الرغم من أن التحليل الطيفي طريقة واقعية من أجل تحديد العناصر الموجودة في الأجسام –مثل النجوم البعيدة، إلا أن العلماء تساءلوا عن سبب قيام كل عنصر بإعطاء خطوط محددة في البداية؟ في العام 1888، اشتق جوهانز ريدبيرغ (Johannes Rydberg) معادلة تصف الخطوط الطيفية الناجمة عن الهيدروجين لكن لم يستطع أي شخص شرح سبب صحة تلك المعادلة. تغير هذا الأمر في العام 1913 عندما طبق نلز بور فرضيات بلانك في التكميم على النموذج "الكوكبي" للذرة والذي وضعه ارنست رذرفورد في العام 1911؛ وينص ذلك النموذج على أن الالكترونات تدور حول النوى بشكل مشابه للطريقة التي تدور فيها الكواكب حول الشمس. وفقاً لموقع physics 2000 (وهو موقع تابع لجامعة كولورادو)، اقترح بور أن الالكترونات مقيدة في مدارات "خاصة" حول النوى الذرية؛ ويمكنها فقط "القفز" بين مدارين محددين والطاقة الناتجة عن القفزة تتسبب بالحصول على لون محدد من الضوء يتم رصده في الخطوط الطيفية. على الرغم من أنّ الخواص المُكممة تم ابتكارها لتكون مجرد خدعة رياضية، إلا أنها فسرت الكثير من الظواهر بحيث أصبحت المبدأ الأساسي لميكانيك الكم. 

\section*{جسيمات الضوء}

في العام 1905، نشر اينشتاين ورقة "وجهة نظر تساعد في فهم إصدارات وانتقالات الضوء"؛ وفيها تصور اينشتاين أن الضوء لا يتحرك على شكل موجة وإنما على شكل "كم طاقة - energy quanta". يُمكن أن تُمتص هذه الحزمة من الطاقة، التي اقترحها اينشتاين، أو تُولد ككل فقط وبشكلٍ خاص عندما تقوم ذرة بالقفز بين حالتي اهتزاز كموميتين. وكما تبين في السنوات اللاحقة، فإن هذا الأمر يُطبق أيضاً عندما يقفز الكترون بين مدارين كموميين. ووفقاً لهذا النموذج، احتوت "كمّات الطاقة" على الفرق في الطاقة الموجود في القفزة بعد تقسيمها على ثابت بلانك؛ ويحدد هذا الفرق في الطاقة لون الضوء الذي يحمله الكم (quanta). بهذه الطريقة لتصور الضوء، قدم اينشتاين رؤى جديدة على سلوك تسع ظواهر مختلفة تشمل الألوان المحددة التي وصف بلانك صدورها عن سلك المصباح. وتشرح هذه الطريقة أيضاً كيف يُمكن لألوان الضوء أيضاً أن تقوم بالتسبب في صدور الالكترونات من سطوح المعادن، وهي ظاهرة تُعرف بالمفعول الكهروضوئي. على أية حال، يقول ستيفن كلاسين (Stephen Klassen)، وهو بروفسور مساعد في الفيزياء من جامعة وينيبج بأن اينشتاين لم يكن موفقاً بالكامل في أخذ هذه الخطوة. في ورقة علمية ظهرت عام 2008 بعنوان "المفعول الكهروضوئي: إعادة تأهيل قصة الفيزياء المدرسية"، يُصرح كلاسين بأن كم الطاقة الذي وضعه اينشتاين لم يكن ضرورياً من أجل شرح الظواهر التسع؛ إذ سيبقى الوصف الموجي للضوء قادراً على شرح ووصف كل من الألوان المحددة التي وجدها بلانك وأيضاً المفعول الكهروضوئي. في الواقع، فإن فوز اينشتاين بجائزة نوبل للفيزياء عام 1921 مثيرٌ للجدل حيث اعترفت لجنة نوبل بـ "اكتشافه لقانون المفعول الكهروضوئي"، الذي لا يعتمد تحديداً على فكرة كم الطاقة. بعد حوالي عقدين على ورقة اينشتاين، تم تعميم تعبير "الفوتون" من أجل وصف كم الطاقة والفضل في ذلك يعود إلى عمل آرثر كومبتون، الذي برهن على أن الضوء المتشتت على الكترون يتغير لونه. برهن هذا العمل أيضاً على أن الضوء (الفوتونات) تتصادم في الواقع مع جسيمات المادة (الالكترونات) وهو ما أكد فرضية اينشتاين. لكن الآن، من الواضح جداً أنه يُمكن للضوء أن يسلك سلوك موجة وجسيم وساهمت "ثنائية موجة-جسيم" الخاصة بالضوء في تأسيس ميكانيك الكم (QM).


\section*{أمواج المادة}

منذ اكتشاف الالكترون في العام 1896، بدأت الأدلة التي تُشير إلى أن المادة موجودة على شكل جسيمات بشكلٍ تدريجي بالتراكم. أدى إثبات العلماء على ثنائية موجة-جسيم في حالة الضوء إلى ظهور السؤال التالي: هل المادة محكومة بالتصرف على أنها جسيمات فقط؟ قد تكون ثنائية موجة-جسيم صحيحة أيضاً بالنسبة للمادة؟ كان الفيزيائي الفرنسي لويس دو بروي أول من قام بمواجهة هذا السؤال. في عام 1924، استخدم دو بروي معادلات اينشتاين في النسبية الخاصة للبرهان على أنه باستطاعة الجسيمات أن تمتلك مميزات موجية ويُمكن للأمواج بالمقابل أن تتمتع بمميزات جسيمية. بعد ذلك في العام 1925، قام عالمان بشكلٍ مستقل باستخدام خطين رياضيين منفصلين واستخدما منطق بروي من أجل شرح كيفية اهتزاز الالكترونات في الذرات (ظاهرة لم يتم شرحها أبداً بالاعتماد على الميكانيك الكلاسيكي). في ألمانيا، قام الفيزيائي فيرنر هايزنبرغ (مع فريق يضم ماكس بورن وباسكوال جوردان) بإنجاز الأمر عبر تطوير "الميكانيك المصفوفي"؛ في حين طور شرودينجر في العام 1926 نظرية مشابهة تُعرف بـ "الميكانيك الموجي". برهن شرودينجر أيضاً أن النهجين متكافئين (على الرغم من أن الفيزيائي السويسري فولفانغ باولي قام بإرسال نتيجة لم تُنشر إلى جوردان وتشرح بأن الميكانيك المصفوفي كان أكثر تعقيداً). حلَّ نموذج هايزنبرغ-شرودينجر الخاص بالذرة، والذي يتصرف فيه كل الكترون كموجة (ويُشار إلى الامر في بعض الأحيان بالسحابة) حول النواة، محل نموذج رذرفورد-بور. وتنص إحدى أشكال النموذج الجديد على أن نهايات الموجة التي يُشكلها الكترون ما يجب أن تتقابل. كتب ميلفن هانا في كتاب "ميكانيكا الكم في الكيمياء، الطبعة الثالثة" (بنجامين 1981) "قيد فرض الشروط الحدية من القيم التي يُمكن أن تأخذها الطاقة". وينتج عن هذا الفرض وجود عدد محدد تماماً من المدارات المسموح بها وهو الأمر الذي يشرح السبب الكامن وراء تكميم بعض الخواص. في نموذج ذرة شرودينجر-هايزنبرغ، يخضع الالكترون إلى "تابع موجي" ويحتل "مدارات فرعية" بدلاً من كونها مدارات رئيسية. في نموذج رذرفورد-بور للذرة، تمتلك المدارات الذرية مجالاً متنوعاً من الأشكال وتتغير من الكرات إلى الدمبل وصولاً إلى شكل النرجس. في العام 1927، طور والتر هيتلر وفريتز لندن الميكانيك الموجي من أجل إثبات كيفية قيام المدارات الذرية بالاجتماع لتشكل مدارات جزيئية، مما فسر بشكلٍ فعال كيفية ارتباط الذرات مع بعضها البعض لتُشكل جزيئات. وكانت تلك مشكلة أخرى لم يتمكن الميكانيك الكلاسيكي من حلها؛ وأدت هذه الرؤى بدورها إلى ظهور "الكيمياء الكمومية".

\section*{مبدأ الارتياب}
 
أيضا في العام 1927، قدم هايزنبرغ إسهاماً رئيسياً آخر في الفيزياء الكمومية. يقول هايزنبرغ بأنه طالما أمكن للمادة أن تسلك سلوك موجة، فبالتالي ستكون بعض الخواص، مثل موضع الالكترون وسرعته، "متتامة"، أي أنه هناك حد (يرتبط بثابت بلانك) يصف مقدار دقة قياس كل خاصية. وفي إطار ما عرف لاحقا بـ "مبدأ الارتياب لهايزنبرغ"، نجد أنه كلما قمنا بمعرفة موقع الالكترون بدقة أكثر، كلما كانت دقة معرفتنا بسرعته أقل، والعكس صحيح. يُطبق مبدأ الارتياب هذا على كل الأجسام التي نُشاهدها في حياتنا اليومية أيضاً، لكن أثره غير ملحوظ بسبب الافتقاد إلى الدقة. ووفقاً لديفيد سلافن من كلية Morningside، إذا ما كانت سرعة كرة سلة معروفة وبخطأ يصل إلى 0.1 ميل في الساعة، سيكون الخطأ المرتكب في تحديد موقع الكرة ستصل إلى 0.000000000000000000000000000008 ميليمتر. 




